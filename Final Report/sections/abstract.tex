\center
\begin{abstract}

     \addcontentsline{toc}{section}{Abstract}

Single-particle reconstruction in \acrfull{cryo-EM}\cite{9016106} is a tool for constructing a 3D model of a biological macromolecule using 2D projections of the macromolecules taken by an electron microscope. An unsupervised classification of the 2D images is required in order to separate macromolecular projections of different conformations. Due to high noise levels and data heterogeneity, sophisticated clustering methods are needed. \\
In our project we employ \acrfull{CD} algorithms to cluster data generated from the \acrfull{MRA} statistical model, the model abstracts away much of the intricacy of \acrshort{cryo-EM} while retaining some of its essential features. \textbf{Fig. \ref{fig:proj_proc}} shows a diagram of the process.\\
In addition to the implementation of \acrshort{CD} algorithms on the \acrshort{MRA} data, we use a modified K-means clustering algorithm to perform the same task, and compare the performance of both methods. In the last section of this document we discuss the differences between the methods and the advantages and disadvantages of the method we propose.

\vspace{50pt}

\begin{figure}[h]
  \centering
  \includegraphics[width=0.8\textwidth]{"figures/Project_Process".png}
  \caption{\textbf{Project process.} Further processing stage presents the idea behind clustering the data and is outside of the scope of the project.}
  \label{fig:proj_proc}
\end{figure}

\end{abstract}

\clearpage