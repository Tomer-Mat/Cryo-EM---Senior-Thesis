\center

\section{Introduction}

\raggedright

Single-particle reconstruction in \acrshort{cryo-EM} is a powerful image-processing tool used to determine the 3D structure of biological macromolecular complexes.
2D images (micrographs) of a macromolecule are taken by an electron microscope and particle picking is performed to extract 2D images of the macromolecules. Essentially, the set of all the 2D images for a given macromolecule spans a 3D model of the macromolecule. Thus, single-particle reconstruction is using the 2D images to build a 3D model of the macromolecule.\\

Due to high sensitivity of the biological macromolecules to radiation damage, electron microscope provides limited electron doses when producing micrographs. This and other factors like low contrast of micrographs and digitalization of the images result in \acrshort{cryo-EM} data having very low \acrfull{SNR}\cite{9016106}.\\

The \acrshort{cryo-EM} technology has the potential to offer the ability to analyze different functional and conformational states of macromolecules, an important ability for the field of molecular biology. Practically, it entails the classification of heterogeneous \acrshort{cryo-EM} data.\\

Many different approaches for \acrshort{cryo-EM} data classification have been developed. Typically, likelihood optimization algorithms and Bayesian inference frameworks are used to deal with data heterogeneity\cite{sigworth1998maximum,scheres2005fast,scheres2014beam,song2013flexibility,chowdhury2015structural}.\\ 

In our project we propose a different approach for \acrshort{cryo-EM} data classification using \acrfull{CD} algorithms that are common in the field of complex networks. According to our approach, data will be classified following the steps:
\begin{itemize}
	\item Converting data into a weighted graph, where each node corresponds to a sample and the edges between nodes represent the degree of similarity between samples.
	\item Applying \acrlong{CD} algorithms to partition the graph into distinct communities
	\item Each community represent a single conformation of a sampled macromolecule.
\end{itemize}

For the sake of an abstraction of the \acrshort{cryo-EM} data we use the Heterogeneous \acrfull{MRA} statistical model, throughout our project we use the simplified 1D version of the model.\\
In our project we introduce the modified KMeans algorithm that takes into account the characteristics of the \acrshort{MRA} data. The modified KMeans is used as a reference point against which \acrfull{CD} algorithms are evaluated. 